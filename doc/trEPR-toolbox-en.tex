%% refart template
%% User-defined template for using refart for documentation purposes
\documentclass[a4paper]{refrep}
\usepackage{textcomp}
\usepackage[utf8]{inputenc}


\newcommand{\matlab}{\textsf{MATLAB$^{\mbox{\scriptsize\textregistered}}$}}
\newcommand{\mathworks}{\textsf{The MathWorks}}
\newcommand{\octave}{\textsf{GNU Octave}}
\newcommand{\gnuplot}{\textsf{gnuplot}}
\newcommand{\macOSX}{\textsf{Mac OS X 10.4}}

\newcommand{\mLaTeXtb}{\textsf{Matlab--LaTeX--Toolbox}}
\newcommand{\trEPRtb}{\textsf{trEPR--Toolbox}}

\newcommand{\fsczwei}{\textsf{fsc2}}
\newcommand{\transient}{\textsf{transient}}
\newcommand{\speman}{\textsf{Speman}}

%\newcommand{\code}[1]{\texttt{#1}}

\newcommand{\bibTeX}{\textsc{bib}\TeX}

\newcommand{\metaGen}{\textsf{metaGen}}

\newenvironment{note}{\vspace*{2\parsep}\hrule\par\textsf{\textbf{NOTE}}}{\par\vspace*{\parsep}\hrule\par\vspace*{2\parsep}}


\title{\trEPRtb\\[1em] {\large for use with\\ \matlab\ and \octave}\\[1em] User's
Guide}
\author{Till Biskup\thanks{till.biskup@physik.fu-berlin.de}}
\date{$ $Id$ $}

% Setze die Tiefe der im Inhaltsverzeichnis erscheinenden Überschriften
% 1 = nur \section-Ebene
% 2 = \section und \section
\setcounter{tocdepth}{1}

\pagestyle{footings}

\begin{document}
\maketitle
\thispagestyle{empty}


\tableofcontents


\chapter{Preface}

This chapter provides an overview of the \trEPRtb, as well as information about
this documentation. The sections are as follows.

\vspace*{3\parsep}

\begin{description}\descriptioncolonfalse
  \item[What is the \trEPRtb?] The toolbox and the kinds of tasks it can perform
  \item[Related Products] \mathworks\ products and products of the author of the
  toolbox related to this toolbox
  \item[Using This Guide] An overview of this guide
  \item[Installation Information] How to determine whether the toolbox is
  installed on your system.
  \item[Typographical conventions] Typographical conventions that this guide uses
\end{description}

\clearpage


\section{What Is the \trEPRtb?}

The \trEPRtb\ is a collection of graphical user interfaces (GUIs) and M--file
functions built on the \matlab\ technical computing environment, but in most
parts compatible to \octave. The toolbox provides you with these main features:

\begin{itemize}
  \item Importing data in different formats saved by the measurement programs 
  \fsczwei\ and \transient.
  \item Preprocessing the raw data (e.g., offset compensation, drift 
  compensation, accumulation).
  \item Viewing the raw and preprocessed data (2D and 1D plots of different 
  kinds).
  \item Analyzing data (e.g. SNR).
  \item Postprocessing data (e.g., saving to different file formats, saving 
  figures).
\end{itemize}


\subsection{Exploring the Toolbox}

The \trEPRtb\ consists of two different environments: a graphical user interface
(GUI) environment and the usual \matlab/\octave\ command line environment.

You can explore the graphical environment by typing

\begin{example}
\ldots 
\end{example}

Click the GUI Help button to learn how to proceed. Additionally, you can follow
the examples in the tutorial sections of this guide that are GUI oriented.

To explore the command line environment, you can list the toolbox functions by
typing 

\begin{example} 
  help trEPRtoolbox
\end{example}

To view the code for any function, type

\begin{example}
  type function\_name 
\end{example}

To view the help for any function, type

\begin{example}
  help function\_name 
\end{example}

You can change the way any toolbox function works by copying and renaming the
M--file, and then modifying your copy.

You can also extend the toolbox by adding your own M--files, or by using it in
combination with other products and toolboxes.


\section{Related Products}

The \trEPRtb\ requires \matlab\ or \octave. Additionally, both, the author of
this toolbox and \mathworks\ provide several related products that are
especially relevant to the kinds of tasks you can perform with the \trEPRtb. For
more information about any of these products, see either

\begin{itemize}
  \item The online documentation for that product if it is installed
  \item The Web site of the author of that toolbox, at
  
  \texttt{http://physik.fu-berlin.de/~biskup/}; see the ``toolbox'' section
  \item \mathworks Web site, at \texttt{http://www.mathworks.com/}; see the
  ``products'' section
\end{itemize}

\begin{note}
The toolboxes listed below all include functions that extend the capabilities of
\matlab/\octave.
\end{note}


\renewcommand\arraystretch{1.5}
\begin{tabular*}{\textwidth}{l@{\extracolsep{\fill}}p{.6\textwidth}}
\hline
\textsf{\textbf{Product}}
&
\textsf{\textbf{Description}}
\\
\hline
Optimization Toolbox 
&
Solve standard and large--scale optimization problems
\\
\mLaTeXtb 
&
Write \LaTeX\ output from within \matlab\ or \octave
\\
\hline
\end{tabular*}


\section{Using This Guide}

\subsection{Expected Background}

This guide assumes that you already have background knowledge in the subject of
EPR and especially transient EPR. If you do not yet have this background, then
you should read an introductory text to EPR, some of which are listed on page
\ldots. 

When learning to use the toolbox, you should start with Chapter
\ref{ch:gettingStarted}, ``Getting started with the \trEPRtb'', which
illustrates the major toolbox features.

More detailed information about the features duscussed in this chapter is
available elsewhere in the guide, and you should browse the documentation
according to your needs.


\subsection{How This Guide Is Organized}

The organization of this guide is described below.


\begin{tabular*}{\textwidth}{p{.35\textwidth}@{\extracolsep{\fill}}p{.55\textwidth}}
\hline
\textsf{\textbf{Chapter}}
&
\textsf{\textbf{Description}}
\\
\hline
Getting Started with the \trEPRtb 
&
Describes a particular example in detail to help you get started with the toolbox.
\\
\ldots 
&
\ldots
\\
Function Reference 
&
Describes all toolbox functions in detail.
\\
\hline
\end{tabular*}


\subsection{Documentaion Examples And Data Sets}

To learn how to use the \trEPRtb, you can follow the examples included in this
guide. A quick way to locate these examples is with the example index, which you
can access via the Help browser.

Some examples use data that is generated as part of the example, while other
examples use data sets that are included with the toolbox. These data seets are
stored as DAT--files and are listed below.


\begin{tabular*}{\textwidth}{p{.2\textwidth}@{\extracolsep{\fill}}p{.7\textwidth}}
\hline
\textsf{\textbf{Data Set}}
&
\textsf{\textbf{Description}}
\\
\hline
\texttt{file1} 
&
description
\\
\texttt{file2} 
&
description
\\
\hline
\end{tabular*}


\section{Installation Information}

To determine if the \trEPRtb\ is installed on your system, type

\begin{example}
  ver 
\end{example}

at the \matlab\ prompt. \matlab\ displays information about the version of
\matlab\ you are running, including a list of installed add-on products and
their version numbers. Check the list to see if the \trEPRtb\ appears.

For information about installing the toolbox, refer to the \matlab\ Installation
 Guide for your platform. If you experience installation difficulties and have 
Web access, look for the installation information at \mathworks\ Web site 
\texttt{http://www.mathworks.com/support}.


\pagebreak

\section{Typographical Conventions}

This guide uses some or all of these conventions.

\maxipagerulefalse
\begin{maxipage}

\begin{tabular*}{\textwidth}{p{.275\textwidth}@{\extracolsep{\fill}}p{.29\textwidth}p{.35\textwidth}}
\hline
\textsf{\textbf{Item}}
&
\textsf{\textbf{Convention}}
&
\textsf{\textbf{Example}}
\\
\hline
example code 
&
\texttt{Monospace} font
&
To assign the value 5 to A, enter
\begin{example}
A = 5 
\end{example}
\\[-2\parsep]
Function names, syntax, filenames, directory/folder names, and user input
&
\texttt{Monospace} font
&
The \texttt{cos} function finds the cosine of each array element.
\linebreak
Syntax line example is
\begin{example}
MLGetVar ML\_var\_name 
\end{example}
\\[-2\parsep]
Buttons and keys
&
\textbf{Boldface} with book title caps
&
Press the \textbf{Enter} key.
\\
Literal strings (in syntax descriptions in reference chapters)
&
\textbf{\texttt{Monospace bold}} for literals
&
\texttt{f = freqspace(n,'\textbf{whole}')}
\\
Mathematical expressions
&
\textit{Italics} for variables
\linebreak
Standard text font for functions, operators, and constants
&
This vector represents the polynomial $p = x^2 + 2x + 3$.
\\
\matlab\ output
&
\texttt{Monospace} font
&
\matlab\ responds with
\begin{example}
A = \linebreak
\quad 5
\end{example}
\\[-2\parsep]
Menu and dialog box titles
&
\textbf{Boldface} with book title caps
&
Choose the \textbf{File Options} menu.
\\
New terms and for emphasis
&
\emph{Italics}
&
An \emph{array} is an ordered collection of information.
\\
Omitted input arguments
&
(\ldots) ellipsis denotes all of the input/output arguments from preceeding syntaxes.
&
\texttt{[c,ia,ib] = union(\ldots)}
\\
String variables (from a finite list)
&
\texttt{\textsl{Monospace italics}}
&
\texttt{sysc = d2c(sysd,'\textsl{method}')}
\\
\hline
\end{tabular*}

\end{maxipage}
\maxipageruletrue


\chapter{Getting Startet with the \trEPRtb}
\label{ch:gettingStarted}

This chapter describes a particular example in detail to help you get started
with the \trEPRtb. In this example, you will\ldots

In doing so, the basic steps involved in any \ldots scenario are illustrated.
These steps include

\vspace*{3\parsep}

\begin{description}\descriptioncolonfalse
  \item[Step 1] Description of step 1 
  \item[Step 2] Description of step 2
  \item[Step 3] Description of step 3 
  \item[Step 4] Description of step 4 
\end{description}

\clearpage


\section{\ldots}


\chapter{Importing, Viewing, and Preprocessing Data}

This chapter describes how to import, view, and preprocess data with the 
\trEPRtb.

The sections are as follows.

\vspace*{3\parsep}

\begin{description}\descriptioncolonfalse
  \item[Importing Data Sets] Select workspace variables that compose the data
  set, list all imported and generated data sets, and delete one or more data
  sets.
  \item[Viewing Data] View the data graphically.
  \item[Preprocessing Data] Compensate the offset, drift, etcetera, accumulate
  several measurements, \ldots
\end{description}

\clearpage


\section{Importing Data Sets}

\begin{description}
  \item[\texttt{read\_fsc2\_file}] Description of the function
  \item[\texttt{ascii\_read\_2Dspectrum}] Description of the function
  \item[\texttt{ascii\_read\_spectrum}] Description of the function
  \item[\texttt{ascii\_read\_timeslice}] Description of the function
  \item[\texttt{trEPR\_read}]
  \item[\texttt{trEPR\_read\_fsc2\_file}]
  \item[\texttt{spemanAscii2trEPR2Ddata}]
\end{description}


\section{Viewing Data}

\begin{description}
  \item[\texttt{B0\_spectrum}] Description of the function
  \item[\texttt{make\_compensated\_2D\_plot}] Description of the function
  \item[\texttt{make\_reduced\_2D\_plot}] Description of the function
  \item[\texttt{trEPR\_timeAxis}] Description of the function
\end{description}


\section{Preprocessing Data}

\begin{description}
  \item[\texttt{reduce\_datapoints\_1D}] Description of the function
  \item[\texttt{running\_average}] Description of the function
  \item[\texttt{trEPR\_compensate\_baseline}] Description of the function
  \item[\texttt{trEPR\_compensate\_drift}] Description of the function
  \item[\texttt{trEPR\_compensate\_frequency}] Description of the function
  \item[\texttt{trEPR\_compensate\_timeslice}] Description of the function
  \item[\texttt{trEPR\_cut\_spectrum}] Description of the function
  \item[\texttt{trEPR\_find\_maximum\_amplitude}] Description of the function
  \item[\texttt{trEPR\_integrated\_bl\_comp}] Description of the function
\end{description}


\chapter{Analyzing Data}

This chapter describes how to analyze data with the 
\trEPRtb.

The sections are as follows.

\vspace*{3\parsep}

\begin{description}\descriptioncolonfalse
  \item[\ldots] Short description.
\end{description}

\clearpage


\section{\ldots}

\begin{description}
  \item[\texttt{SNR\_analysis}] Description of the function
  \item[\texttt{trEPR\_snr}] Description of the function
  \item[\texttt{svd\_analysis}] Description of the function
  \item[\texttt{trEPR\_expfit\_timeslice}] Description of the function
\end{description}


\chapter{Postprocessing Data}

This chapter describes how to postprocess data with the 
\trEPRtb.

The sections are as follows.

\vspace*{3\parsep}

\begin{description}\descriptioncolonfalse
  \item[Saving Figures] Short description.
  \item[Saving Data] Short description.
\end{description}

\clearpage


\section{\ldots}

\begin{description}
  \item[\texttt{asciiSaveData}] Description of the function
  \item[\texttt{ascii\_save\_2Dspectrum}] Description of the function
  \item[\texttt{ascii\_save\_spectrum}] Description of the function
  \item[\texttt{ascii\_save\_timeslice}] Description of the function
  \item[\texttt{save\_figure}]
\end{description}


\chapter{Using the GUI}

This chapter describes how to use the graphical users interface (GUI) that comes
with the \trEPRtb.

The sections are as follows.

\vspace*{3\parsep}

\begin{description}\descriptioncolonfalse
  \item[Starting the GUI] Short description.
  \item[Loading Data] Short description.
  \item[Preprocess Data] Short description.
  \item[Viewing Data] Short description.
  \item[Saving Data] Short description.
\end{description}

\clearpage


\section{Starting the GUI}

To invoke the GUI that comes with the \trEPRtb, type

\begin{example}
  trEPRGUI
\end{example}

at the \matlab\ prompt.



\begin{description}
  \item[\texttt{GUI\_fsc2\_display}] Description of the function
  \item[\texttt{trEPR\_fsc2\_display}] Description of the function
\end{description}


\chapter{Function Reference}

This chapter describes the toolbox M--file functions that you use directly. A
number of other M--file helper functions are provided with this toolbox to
support the functions listed below. These helper functions are not documented
because they are not intended for direct use.

\vspace*{3\parsep}

\begin{description}\descriptioncolonfalse
  \item[Functions --- By Category] Contains a series of tables that group
  functions by category.
  \item[Functions --- Alphabetical List] Lists all the functions alphabetically 
\end{description}

\clearpage


\section{Functions --- By Category}

\subsection{Importing Data}

\begin{description}
  \item[\texttt{read\_fsc2\_file}] Description of the function
  \item[\texttt{ascii\_read\_2Dspectrum}] Description of the function
  \item[\texttt{ascii\_read\_spectrum}] Description of the function
  \item[\texttt{ascii\_read\_timeslice}] Description of the function
  \item[\texttt{trEPR\_read}]
  \item[\texttt{trEPR\_read\_fsc2\_file}]
  \item[\texttt{spemanAscii2trEPR2Ddata}]
\end{description}


\subsection{Viewing Data}

\begin{description}
  \item[\texttt{B0\_spectrum}] Description of the function
  \item[\texttt{make\_compensated\_2D\_plot}] Description of the function
  \item[\texttt{make\_reduced\_2D\_plot}] Description of the function
  \item[\texttt{trEPR\_timeAxis}] Description of the function
\end{description}


\subsection{Preprocessing Data}

\begin{description}
  \item[\texttt{reduce\_datapoints\_1D}] Description of the function
  \item[\texttt{running\_average}] Description of the function
  \item[\texttt{trEPR\_compensate\_baseline}] Description of the function
  \item[\texttt{trEPR\_compensate\_drift}] Description of the function
  \item[\texttt{trEPR\_compensate\_frequency}] Description of the function
  \item[\texttt{trEPR\_compensate\_timeslice}] Description of the function
  \item[\texttt{trEPR\_cut\_spectrum}] Description of the function
  \item[\texttt{trEPR\_find\_maximum\_amplitude}] Description of the function
  \item[\texttt{trEPR\_integrated\_bl\_comp}] Description of the function
\end{description}


\subsection{Analyzing Data}

\begin{description}
  \item[\texttt{SNR\_analysis}] Description of the function
  \item[\texttt{trEPR\_snr}] Description of the function
  \item[\texttt{svd\_analysis}] Description of the function
  \item[\texttt{trEPR\_expfit\_timeslice}] Description of the function
\end{description}


\subsection{Postprocessing Data}

\begin{description}
  \item[\texttt{asciiSaveData}] Description of the function
  \item[\texttt{ascii\_save\_2Dspectrum}] Description of the function
  \item[\texttt{ascii\_save\_spectrum}] Description of the function
  \item[\texttt{ascii\_save\_timeslice}] Description of the function
  \item[\texttt{save\_figure}]
\end{description}


\subsection{GUI Functions}

\begin{description}
  \item[\texttt{GUI\_fsc2\_display}] Description of the function
  \item[\texttt{trEPR\_fsc2\_display}] Description of the function
\end{description}


\subsection{General Purpose}

\begin{description}
  \item[\texttt{index}] Description of the function
  \item[\texttt{substr}] Description of the function
  \item[\texttt{start\_logging}] Description of the function
  \item[\texttt{stop\_logging}] Description of the function
  \item[\texttt{get\_file\_basename}]
  \item[\texttt{get\_file\_extension}]
  \item[\texttt{get\_file\_path}]
\end{description}


\subsection{Toolbox Functions}

\begin{description}
  \item[\texttt{trEPRinfo}] Description of the function
  \item[\texttt{trEPRinstall}] Description of the function
%  \item[\texttt{trEPRtoolboxRevision}] Description of the function
\end{description}


\clearpage

\section{Functions --- Alphabetical List}


This section contains detailed descriptions of all toolbox functions. Each
function reference page contains some or all of this information:

\begin{itemize}
  \item The function name
  \item The purpose of the function
  \item The function syntax
  
  All valid input arguments and output argument combinations are shown. In some
  cases, an ellipsis (\ldots) is used for the input arguments. This means that
  all preceding input argument combinations are valid for the specified output
  argument(s).
  
  \item A description of each argument
  \item A description of the function
  \item Additional remarks about usage
  \item An example of usage
  \item Related functions
\end{itemize}


\clearpage

\renewcommand{\subsectionmark}[1]{ 
	\markboth{#1}{#1}
} 

\subsection{Functionname}

\subsubsection{Purpose}

One sentence telling what the function is good for.


\subsubsection{Syntax}

\begin{example}
[ output ] = functionname ( input1, input2, input3, \ldots ) 
\end{example}


\subsubsection{Arguments}

\begin{description}
  \item[\texttt{input1}] Description of the argument 
  \item[\texttt{input2}] Description of the argument 
  \item[\texttt{input3}] Description of the argument 
  \item[\texttt{output}] Description of the argument 
\end{description}


\subsubsection{Description}

\texttt{functionname} does something.


\subsubsection{Remarks}

Some remarks about the function.


\subsubsection{Example}

Some introductory text\ldots

\begin{example}
  [ output ] = functioname ( inputa, inputb, inputc ); 
\end{example}


\subsubsection{See Also}

\texttt{functionNameA, functionNameB}


\clearpage


\subsection{Functionname2}

\subsubsection{Purpose}

One sentence telling what the function is good for.


\subsubsection{Syntax}

\begin{example}
[ output ] = functionname ( input1, input2, input3, \ldots ) 
\end{example}


\subsubsection{Arguments}

\begin{description}
  \item[\texttt{input1}] Description of the argument 
  \item[\texttt{input2}] Description of the argument 
  \item[\texttt{input3}] Description of the argument 
  \item[\texttt{output}] Description of the argument 
\end{description}


\subsubsection{Description}

\texttt{functionname} does something.


\subsubsection{Remarks}

Some remarks about the function.


\subsubsection{Example}

Some introductory text\ldots

\begin{example}
  [ output ] = functioname ( inputa, inputb, inputc ); 
\end{example}


\subsubsection{See Also}

\texttt{functionNameA, functionNameB}


%\clearpage


\end{document}