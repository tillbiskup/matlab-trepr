\documentclass{article}
\usepackage{german}
\usepackage[paper=a4paper,margin=1in]{geometry}
\usepackage[utf8]{inputenc}

\usepackage{textcomp}

\usepackage{natbib}


\def\matlab{\textsf{MATLAB}$^\mbox{\tiny\textregistered}$}
\def\octave{\textsf{GNU Octave}}

\newcommand{\func}[1]{\texttt{#1}}
\newcommand{\cmd}[1]{\texttt{#1}}
\newcommand{\file}[1]{\texttt{#1}}
\newcommand{\var}[1]{\texttt{#1}}

\newenvironment{Quote}{\begin{quote}\small\vspace*{1em}\hrule\vspace*{1ex}
}{\end{quote}\hrule\vspace*{1em}}
%\renewenvironment{quote}{\begin{Quote}}{\end{Quote}}


\title{Unterschiede zwischen \matlab\ und \octave}
\author{Till Biskup}
\date{$ $Revision$ $, $ $Date$ $}


\begin{document}
\maketitle\thispagestyle{empty}

\begin{abstract}
  Aus der Praxis heraus, Funktionen zu implementieren, die sowohl unter 
  \matlab\ als auch unter \octave\ laufen, entstand und entsteht diese
  Zusammenfassung wichtiger Unterschiede zwischen beiden Programmen, angefangen
  von Kleinigkeiten wie Kommentarzeichen bis hin zu größeren
  Funktionalitäts--Bereichen wie die Plot--Routinen.
\end{abstract}

\begin{small}
\tableofcontents 
\end{small}

\section{Grundlegendes}

\octave\ wurde unter anderem mit dem Ziel entwickelt, eine frei verfügbare
Alternative zum kommerziellen \matlab\ zu haben. In weiten Teilen sind beide
Programm--Pakete tatsächlich kompatibel. Allerdings ergeben sich in zahlreichen
Kleinigkeiten Unterschiede, die eine Portierung von Programmen auf die jeweils
ander Plattform und die Parallelentwicklung für beide Plattformen nicht immer
einfach machen.


\section{Kommentare}

\octave\ kennt zwei Kommentarzeichen: ``\%'' und ``\#''. In \matlab\ ist dagegen
nur das ``\%'' als Kommentar erlaubt. Die bei \octave\ häufig zu findenden
doppelten Kommentarzeichen (\%\%) gerade am Beginn eines ``m--files'', die dann
auch beim Aufruf \cmd{help <funktionsname>} ausgegeben werden, erscheinen in
\matlab\ dann mit einem vorangestellten ``\%''--Zeichen.


\section{String--Funktionen}

\matlab\ kennt die Funktionen \cmd{substr} (Extrahiert aus einem String einen 
Teilstring) und \cmd{index} (sucht nach dem Vorkommen eines Stringes in einem 
String und gibt die Position des \emph{ersten Vorkommens} zurück) nicht. Beide
Funktionen wurden im vorliegenden Projekt für \matlab\ implementiert.

\end{document}