\documentclass[a4paper]{refart}
\usepackage{german}
\usepackage[utf8]{inputenc}

\usepackage{amsmath}
\usepackage{amssymb}

\pagestyle{footings}

% Definitionen eigener Befehle
\newcommand{\matlab}{\textsf{MATLAB}$^\mbox{\small\circledR}$}
\newcommand{\octave}{\textsf{GNU Octave}}
\newcommand{\toolbox}{\textsf{trEPR--Toolbox}}
\newcommand{\fsczwei}{\textsf{fsc2}}
\newcommand{\robodoc}{\textsf{ROBODoc}}

\title{Die trEPR--Toolbox für \matlab/\octave}
\author{Till Biskup}
\date{$ $Id$ $}

\begin{document} 
\maketitle

\begin{abstract}
  Die \toolbox\ wurde dazu geschrieben, um mittels transienter EPR gemessene
  Daten aufzuarbeiten und einer anschließenden Auswertung zugänglich zu machen.
  Sie dient werder der Simulation noch der anderweitigen konkreten Auswertung,
  sondern stellt die Spektren dafür zur Verfügung. Die \toolbox\ wurde
  grundsätzlich von einem spezifischen Meßaufbau ausgehend entwickelt, die
  einzelnen Funktionen sind aber allgemein genug gehalten, so daß sie auch mit
  anderen Meßdaten verwendet werden können.
\end{abstract}


\section{Einführung}

\subsection{Grundlegendes zur Toolbox}

\begin{itemize}
  \item modular aufgebaut: pro Aufgabe eine eigene Funktion
  \item grundsätzlich kompatibel zu \octave
  \item jede Funktion dokumentiert
  \item Toolbox liegt in Versionsverwaltung (SVN)\\ D.h. es kann über die
  Logdatei einer Datenaufbereitung jederzeit nachvollzogen werden, mit welcher
  Version einer Funktion die Daten aufbereitet wurden.
\end{itemize}


\subsection{Spezifika des Meßaufbaus}

\begin{itemize}
  \item \fsczwei\ als Meß--Software
\end{itemize}


\section{Datenaufbereitung}

Gemessen werden in der Regel zweidimensionale Datensätze (2D)\footnote{Die
Bezeichnung dieser als 3D--Graph darstellbaren Datensätze als
\emph{zwei}dimensional beruht darauf, daß es sich dabei um Daten handelt, die
von \emph{zwei} Variablen (Magnetfeld, Zeit) abhängt.}, von denen dann jeweils
Spektren in einer Dimension (Magnetfeld oder Zeit) aufgetragen und ausgewertet
werden. Daher läßt sich die Datenaufbereitung ebenfalls in diese beiden Fälle
unterscheiden. Es sei aber schon hier erwähnt, daß die Aufnahme von Zeitkurven
auch von Details bei den Meßparametern her von der Aufnahme über den breiteren
B$_0$--Bereich unterschiedlich ist (höhere Dämpfung der Mikrowelle, sehr
schmaler B$_0$--Bereich).

\subsection{B$_0$--Spektren}

Die Datenaufbereitung der B$_0$--Spektren läßt sich grob in die folgenden Schritte
untergliedern:

\begin{enumerate}
  \item Einlesen der Rohdaten
  \item Offset--Korrektur über das Pretrigger--Signal
  \item Drift--Korrektur
  \item Akkumulation mehrerer (gemessener) Spektren
\end{enumerate}

Für jeden dieser Schritte stehen entsprechende Funktionen zur Verfügung. Der
gesamte Prozeß der Datenaufarbeitung und Akkumulation mehrerer Messungen wird
durch zwei interaktive Programme, die ihrerseits auf diese Funktionen
zurückgreifen, zusätzlich vereinfacht.


\subsubsection{Einlesen der Rohdaten}

Das Meßprogramm \fsczwei\ schreibt die Daten als reine ASCII--Dateien in einer
\matlab--kompatiblen Weise, so daß sie grundsätzlich sehr einfach in \matlab\
eingelesen werden können. Jede dieser Daten--Dateien enthält neben den reinen
Meßdaten in \emph{einer} einzigen Spalte als Dateikopf das gesamte
\fsczwei--Programm, mit dem gemessen wurde, in einem \matlab--Kommentar und an
dessen Ende eine Zusammenfassung der wichtigsten Meßparameter.

Eine mit \fsczwei\ geschriebene Datei läßt sich leicht an der UNIX--typischen
ersten Zeile des als Kommentar am Beginn der Datei enthaltenen
\fsczwei--Skriptes\footnote{der sogenannten ``shebang'', auch ``hashbang'',
bestehend aus der Zeichenfolge \texttt{\#!}, gefolgt von dem Pfad zum
Interpreter des Skriptes, der aufgerufen werden soll, um das Skript abzuarbeiten} 
erkennen: Die erste nichtleere Zeile beginnt (vom \matlab--Kommentarzeichen '\%'
einmal abgesehen) mit der Zeichenfolge

\begin{verbatim}
#!/usr/local/bin/fsc2
\end{verbatim}

Das wird zur Erkennung des Dateityps genutzt. Anschließend liest die
entsprechende \matlab--Funktion aus dem Kommentarkopf der Datei die wichtigen
Meßparameter aus und gibt diese als weitere Parameter zusammen mit den
eigentlichen Daten (als NxM--Matrix) zurück.


\subsubsection{Offset--Korrektur}

Die einzelnen aufgenommenen Zeitkurven weisen kleine Unterschiede in der Lage
ihrer Null--Linie auf. Da die Daten allerdings so aufgenommen werden, daß ein
kleiner Teil der Zeitkurve vor dem Trigger (Laser--Puls) liegt, kann dieser
Bereich des Signals (Pretrigger--Signal) verwendet werden, um das Offset der
einzelnen Zeitkurven herauszurechnen. Dazu wird der Mittelwert des Rauschens des
Pretrigger--Signals jeder einzelnen Zeitkurve auf Null gesetzt.


\subsubsection{Drift--Korrektur}

Aus verschiedenen Gründen (bei Radikalpaarsignalen das darunterliegende
Triplett--Signal, allgemein der exponentielle Zerfall der Signalintensität durch
die fortwirkende Einwirkung des Lasers auf die Probe während einer Messung)
können die Spektren in Richtung des Magnet--Feldes eine Drift aufweisen.
Außerdem kommt es durch den Laserblitz zu einer kurzzeitigen Erwärmung des
Resonators und damit zu einer Störung des Systems, die sich in einer
Signalveränderung (Signalanstieg direkt nach dem Laserpuls) niederschlägt.

Um beide Effekte aus den vorliegenden, offset--korrigierten Daten
herauszurechnen, wird über eine kleine Zahl von Zeitkurven zu Beginn und am Ende
des Spektrums (im sogenannten offresonanten Bereich, da hier keine
EPR--Übergänge in der Probe in Resonanz sind) gemittelt und die gewichtete Summe
dieser beiden gemittelten Zeitkurven von jeder einzelnen Zeitkurve abgezogen.

Für die Wichtung kommt die Drift des Spektrums in Magnetfeld--Richtung zur
Verwendung, die durch einen Polynomfit über die letzten zehn B$_0$--Spektren in
der Zeitdomäne (da, wo das EPR--Signal in aller Regel schon abgefallen ist)
errechnet wird. Auch wenn für das Anfitten dieser Drift Polynome der ersten bis
siebten Ordnung zur Verfügung stehen, reicht in der Regel ein Fit erster Ordnung
aus.


\subsubsection{Akkumulation}

Zur Verbesserung des Signal--Rausch--Abstandes (SNR) ist es meist notwendig,
mehrere Messungen aufzuakkumulieren. Diese Akkumulation erfolgt in zwei
Schritten:
 
\begin{enumerate}
  \item Korrektur der Mikrowellen--Frequenz
  \item gewichtete Akkumulation beider Spektren
\end{enumerate}

Die \marginlabel{Frequenz--Korrektur} Korrektur der MW--Frequenz ist notwendig,
da einzelne Messungen meist nicht bei exakt derselben MW--Frequenz aufgenommen
wurden und die Lage des Spektrums (um $g=2$) sowohl von der MW--Frequenz als auch vom
Magnetfeld abhängt:

\begin{align*} 
g &= \frac{h\nu}{\mu_{\beta}B_0}
\end{align*}

Hinzu kommt eine Ungenauigkeit in der Bestimmung des Magnetfeldes bei dem
gegebenen Meßaufbau. Die Korrektur der Frequenz erfolgt durch manuelles
Verschieben der beiden zu akkumulierenden Spektren gegeneinander.

Die \marginlabel{gewichtete Akkumulation} Akkumulation der Spektren erfolgt
nicht als einfache Aufsummierung, sondern gewichtet, so daß die Qualität des
resultierenden Spektrums maximal wird.

\begin{align*}
  \text{Spectrum}_\text{acc} &= \text{Spectrum}_1 + A \; \text{Spectrum}_2
\end{align*}

Dazu wird die Qualität der beiden Spektren für einen definierten Bereich des 
Wichtungsfaktors ($A = 0.1 \dots 10$) bestimmt.

Die Qualität eines Spektrums wird aus dem Verhältnis seiner maximalen Amplitude
zur Standardabweichung des offresonanten Rauschens, berechnet aus einer
festlegbaren Zahl von Datenpunkten, bestimmt.


\subsection{Zeitkurven}




\section{Die einzelnen Funktionen der Toolbox}

\subsection{Grundlegende Funktionen}

\subsubsection{Dateihandhabung}

...\marginlabel{Dateiname}

...\marginlabel{Dateiendung}

...\marginlabel{Dateipfad}


\subsubsection{Logging}

...\marginlabel{start\_logging}

...\marginlabel{stop\_logging}


\subsubsection{String--Operationen}

...\marginlabel{index}

...\marginlabel{substr}


\subsection{Funktionen zum Umgang mit Dateien}

\subsubsection{fsc2--Dateien}


\subsubsection{B$_0$--Spektren (1D und 2D)}


\subsubsection{Zeitkurven}


\subsection{Datenaufbereitung in 2D}


\subsection{Datenaufbereitung in 1D}


\subsection{Interaktive Programme}


\begin{appendix}

\begin{fullpage}

\section{Dokumentation der einzelnen Funktionen}

Die folgende Dokumentation wurde mittels \robodoc\ automatisch direkt aus den 
Kommentaren der einzelnen Dateien erstellt und an dieser Stelle eingebunden. 
Aus diesem Grund ist die Dokumentation in englischer Sprache verfaßt.


\scriptsize
\input{ROBODoc}

\end{fullpage}
\end{appendix}

\end{document}