\documentclass{article}
\usepackage{german}
\usepackage[paper=a4paper,margin=1in]{geometry} \usepackage[utf8]{inputenc}

\usepackage{textcomp}

\usepackage{natbib}

\usepackage{amsmath}

\newcommand{\vect}[1]{\boldsymbol{\mathrm #1}}

\title{trEPR--Messungen: Konzeption der Auswertungsroutinen}
\author{Till Biskup}
\date{\today, $ $Revision$ $}


\def\matlab{\textsf{MATLAB}$^\mbox{\tiny\textregistered}$} 
\def\octave{\textsf{GNU Octave}} \def\robodoc{\textsf{ROBODoc}} 
\def\fscii{\texttt{fsc2}} \def\eclipse{\textsf{eclipse}}

\newcommand{\func}[1]{\texttt{#1}}
\newcommand{\cmd}[1]{\texttt{#1}}
\newcommand{\file}[1]{\texttt{#1}}
\newcommand{\var}[1]{\texttt{#1}}

\newenvironment{Quote}{\begin{quote}\small\vspace*{1em}\hrule\vspace*{1ex}
}{\end{quote}\hrule\vspace*{1em}}
%\renewenvironment{quote}{\begin{Quote}}{\end{Quote}}

\begin{document}
\maketitle\thispagestyle{empty}

\begin{abstract}
  Die folgenden Ausführungen sind eine kurze Zusammenfassung der Konzeption 
  eines Auswertungsprogramms für die Daten aus transienten EPR--Messungen. 
  Zunächst werden die grundlegenden Ideen für die Umsetzung kurz aufgeführt, 
  anschließend die einzelnen Programmblöcke detaillierter erklärt. Ausgehend 
  von dieser Dokumentation sollte es möglich sein, ein entsprechendes Programm 
  in \matlab\ zu erstellen.
\end{abstract}

\begin{small}
\tableofcontents
\end{small}

\section{Grundsätzliches}

\begin{itemize}
  \item Aufteilung der einzelnen Programmblöcke in einzelne \matlab--Funktionen
  \item Dokumentation jeder einzelnen Funktion (am besten mit \robodoc) 
  \emph{in Englisch}
  \item Interaktive Abfrage von Parametern nicht in den Kernfunktionen, sondern 
  in eigenen ``Treiber--Funktionen'' bzw. in einem ```Meta--Skript'', das die 
  einzelnen Funktionen in der richtigen Reihenfolge aufruft
  \item Logging aller Einzelschritte und wichtigen Werte in einer Log--Datei 
  (ähnlich einer \textsf{gaussian}--Aus\-gabedatei)\\ Kann in einem ersten 
  Schritt mit der Funktion \texttt{diary} von \matlab/\octave\ realisiert 
  werden (vorausgesetzt im ``Meta--Skript'' sind entsprechende zusätzliche 
  Ausgaben programmiert).
  \item Versionsverwaltung (CVS bzw. Subversion) der einzelnen Skripte\\ Die 
  jeweils aktuelle Fassung aller Dateien (inkl. Dokumentation) liegt auf einer 
  Homepage im Internet. Die eigentliche Entwicklungsarbeit an den 
  Kernfunktionen (und damit gleichzeitig mit der Versionsverwaltung) beschränkt 
  sich auf einen kleinen Benutzerkreis.
  \item ``Meta--Skript'', das auf die einzelnen Funktionen zurückgreift und 
  einen Datensatz komplett analysiert (später durch --- in \matlab\ 
  programmierte --- GUI ersetzbar)
  \item Nach Möglichkeit für die nichtgrafischen Teile (alle Kernfunktionen) 
  Kompatibilität zu \octave. Grund: \octave\ ist im Gegensatz zu \matlab\ freie 
  Software und ohne Probleme jederzeit unter jedem Betriebssystem (Unix/Linux, 
  Mac, Windows) einsetzbar.
\end{itemize}


\section{Beschreibung der Kernfunktionen}

Die Funktionalität wird nicht wie bisher durch ein Gesamtprogramm 
bereitgestellt, sondern durch mehrere kleine Funktionen (``Kernfunktionen''), 
die jeweils eine einzige Aufgabe erfüllen. Diese Kernfunktionen erhalten alle 
notwendigen Parameter durch ihren Aufruf, sie besitzen selbst keine Strukturen, 
die interaktiv nach solchen Parametern fragen.

Ziel der Beschreibung ist neben der Darstellung der einzelnen Funktionsbereiche 
und der Aufteilung die Darlegung des allgemeinen Vorgehens bei der Datenanalyse.


\subsection{Rohdaten einlesen}

\begin{description}
  \item[Name] read\_fsc2\_data
  \item[Funktion] Liest die vom Spektrometer kommenden Rohdaten (1D) ein und 
  wandelt sie in ein 2D--Array um.
  \item[Übergabe-Parameter] Dateiname
  \item[Rückgabewert] 2D--Array
\end{description}

Die Spektrometer--Software (\fscii) schreibt an den Beginn der Daten--Datei 
zunächst das Skript, mit dem die Daten aufgenommen wurden (durch 
``\%''--Zeichen auskommentiert für \matlab) und darunter in einer einzigen 
Spalte (1D) alle aufgezeichneten Werte.

Gemessen wird für jeden $B_0$--Wert eine Zeitkurve mit einer definierten Zahl 
von Werten. Die Zahl der Werte pro Zeitkurve ($t$--Dimension), der Bereich des 
durchfahrenen $B_0$--Feldes und dessen Schrittweite ($B_0$--Dimension) werden 
am Ende des auskommentierten Bereiches vor den eigentlichen Daten ausgegeben. 
Sinnvollerweise werden diese Daten nicht per Hand eingegeben, sondern von hier 
ausgelesen.

Im Folgenden ist ein Ausschnitt einer solchen Datendatei wiedergegeben. Sehr 
zur Vereinfachung des Auslesens trägt die Kennzeichnung der einzelnen Parameter 
durch die vorangestellten Bezeichner bei. Unterhalb der Parameter erscheint der 
vor der eigentlichen Messung vom Benutzer eingegebene Kommentar.

\begin{Quote}
\begin{verbatim}
% 
% Number of runs     = 1
% Start field        = 3410.00000 G
% End field          = 3510.00000 G
% Field step width   = 0.500000000 G
% Sensitivity        = 5.00000000 mV/div
% Number of averages = 20
% Time base          = 2.00000000 us/div
% Number of points   = 500
% Trigger position   = 50
% Slice length       = 20.0000000 us
% 
% 9,68455 GHz
% 20 dB
% 274 K
% 460 nm (1/8)
% 
0.000650009918
\end{verbatim}
\end{Quote}



\subsection{Reihenfolge anpassen}

\begin{description}
  \item[Name] read\_fsc2\_data
  \item[Funktion] Ordnet die Daten (2D--Array) nach aufsteigendem $B_0$--Wert an
  \item[Übergabe-Parameter] 2D--Array
  \item[Rückgabewert] 2D--Array
\end{description}

Gemessen wird meist symmetrisch: Einmal wird das $B_0$--Feld in aufsteigender, 
einmal in absteigender Richtung durchfahren. Zur weiteren Verarbeitung ist es 
aber notwendig, daß alle Datensätze in der gleichen Sortierung vorliegen.


\subsection{Akkumulation}

\begin{description}
  \item[Name] accumulate\_measurements
  \item[Funktion] Aufakkumulation mehrerer Messungen
  \item[Übergabe-Parameter] mehrere 2D--Arrays mit Daten
  \item[Rückgabewert] 2D--Array mit aufakkumulierten Daten
\end{description}

Zur Verbesserung des Signal--Rausch--Abstandes werden mehrere Messungen 
aufakkumuliert.


\subsection{Offset--Korrektur über Pretrigger--Signal}

\begin{description}
  \item[Name] pretrigger\_offset
  \item[Funktion] Offset--Korrektur des Signals über die Pretrigger--Zeit
  \item[Übergabe-Parameter] 2D--Array mit Daten, Pretrigger-Zeit
  \item[Rückgabewert] 2D--Array, offset--korrigiert
\end{description}

Langsame Oszillationen oder Drifts (Temperaturschwankungen, Erschütterungen, 
etc.) werden durch Abziehen eines Offsets korrigiert. Das Offset wird über die 
vor dem Laserpuls aufgenommenen Datenpunkte der Zeitkurve (deren Anzahl steht 
am Ende des einleitenden Kommentars der Datendatei) ermittelt.


\subsection{Integration eines 1D--Spektrums ($B_0$)}

\begin{description}
  \item[Name] 1D\_B0\_spectrum
  \item[Funktion] Optimale Integration eines 1D--$B_0$--Spektrums zu einem 
  definierten Zeitpunkt $t$
  \item[Übergabe-Parameter] 2D--Array mit Daten, Zeitpunkt $t$, 
  Integrationsbereich
  \item[Rückgabewert] integriertes 1D--Spektrum
\end{description}

Oft werden nicht 2D--Spektren, sondern 1D--$B_0$--Spektren zu festen Zeiten $t$ 
ausgewertet. Für einen verbesserten Signal--Rausch--Abstand wird rund um einen 
gegebenen Zeitpunkt $t$ das Signal aufintegriert. Es gibt ein optimales 
Integrationsintervall innerhalb des vorgegebenen Integrationsbereiches, das den 
Signal--Rausch--Abstand maximiert. Die Funktion optimiert die Integration unter 
den gegebenen Parametern auf dieses optimale Integrationsintervall und gibt das 
geglättete 1D--Spektrum zurück


\subsection{$B_0$--Drift--Korrektur}

Häufig zeigen die Spektren eine Drift über die gesamte Messung. Das ist auch 
der Grund für die symmetrische Aufnahme (s.o.). Die Korrektur dieser Drift ist 
nichttrivial, da sie normalerweise erst zeitlich nach dem Trigger--Puls 
auftritt (das Pretrigger--Offset driftet nicht). Daher werden im folgenden zwei 
Routinen vorgestellt, die diese Drift entlang beider Dimensionen ($B_0$ und 
$t$) korrigieren.

\subsubsection{$B_0$--Drift untersuchen}

\begin{description}
  \item[Name] drift\_evaluation
  \item[Funktion] Korrektur der $B_0$--Drift entlang der $t$--Achse
  \item[Übergabe-Parameter] 2D--Array, Zahl der $B_0$--Kurven zur Bestimmung 
  der Drift
  \item[Rückgabewert] Drift (gemittelt über die Zahl der $B_0$--Kurven), 
  Exponenten des Polynom--Fits 1. Ordnung, Werte des Polynom--Fits 1. Ordnung, 
  Exponenten des Polynom--Fits 2. Ordnung, Werte des Polynom--Fits 2. Ordnung
\end{description}

Um die Drift als solche zu charakterisieren, wird das $B_0$--Signal über die 
letzten Zeitpunkte gemittelt und linear bzw. quadratisch gefittet. Der jeweils 
passendere Fit kann dann verwendet werden, um in einem nächsten Schritt die 
Drift zu korrigieren.


\subsubsection{$B_0$--Drift--Korrektur entlang $t$}

\begin{description}
  \item[Name] drift\_compensation\_along\_t
  \item[Funktion] Korrektur der $B_0$--Drift entlang der $t$--Achse
  \item[Übergabe-Parameter] 2D--Array, Trigger--Position, $t >> t_1$, Zahl der 
  Off--Resonance--Zeitkurven zur Bestimmung von $t_1$, Polynom--Fit des Driftes
  \item[Rückgabewert] 2D--Array, drift--korrigiert
\end{description}

Das gefittete Polynom (linear oder quadratisch) wird für alle Zeitpunkte vom 
Signal abgezogen. Dabei tritt das Problem auf, daß das (bereits korrigierte, 
driftfreie) Pretrigger--Offset wieder mit einer Drift überlagert wird. Die 
Lösung für dieses Problem liegt in einer entlang $t$ gewichteten 
Drift--Korrektur.

Entlang der $t$--Achse lassen sich (für einen Bereich von $B_0$, in dem kein 
Signal beobachtet wird) drei Bereiche unterscheiden:
\begin{enumerate}
  \item Zeit vor dem Trigger--Puls $t_0$
  \item Zeit zwischen dem Trigger--Puls $t_0$ und dem Erreichen des (stabilen) 
  Wertes nach dem Trigger (Zeitpunkt $t_1$)
  \item Zeit zwischen $t_1$ und $t$ (dem Ende der $t$--Achse)
\end{enumerate}

Zur entlang $t$ gewichteten Drift--Korrektur muß in einem ersten Schritt der 
Zeitpunkt $t_1$ bestimmt werden, anschließend können die Vektoren mit den 
Gewichten erstellt werden, die dann für die Drift--Korrektur zum Einsatz kommen.

Das Intervall zwischen $t_0$ (Trigger--Impuls) und $t_1$ sollte möglichst kurz 
gehalten werden, um das Signal, das zügig nach $t_0$ einsetzt, möglichst nicht 
zu beeinflussen.

Zur Bestimmung von $t_1$ wird folgendes Vorgehen vorgeschlagen:
\begin{enumerate}
  \item Berechnung des (arithmetischen) Mittels und der Standardabweichung für 
  das Signal deutlich nach dem Trigger bis zum Ende der $t$--Achse.
%  \item Berechnung des (arithmetischen) Mittels vom Offset
  \item Inkrementeller Vergleich des Signals entlang $t$ ab $t_0$, bis es dem 
  Wert für das (stabile) Signal nach dem Trigger entspricht (als Referenz für 
  das stabile Signal wird der um die Standardabweichung des stabilen Signals 
  verringerte Mittelwert des stabilen Signals verwendet).
  \item Wenn das Signal $\geq$ der Referenz ist, ist der Zeitpunkt $t_1$ 
  gefunden.
\end{enumerate}

In einem zweiten Schritt kann der Gewichts--Vektor (\var{drift\_weights}) 
erstellt werden. Grundsätzlich gilt:

\begin{align*}
  \text{\var{drift\_weights}} &= 1 & \forall & \quad t=0 \leq t \leq t_0\\ 
  \text{\var{drift\_weights}} &= 1\dots 0 & \forall & \quad t_0 < t \leq t_1\\
  \text{\var{drift\_weights}} &= 0 & \forall & \quad t_1 < t \leq t_\text{max}
\end{align*}

Für den Bereich zwischen $t_0$ und $t_1$ wird ein linearer Anstieg 
(\var{slope\_of\_weights}) angenommen:

\begin{Quote}
\begin{verbatim}
  slope_of_weights = 1 / ( t1 - t0 )
  
  for i = ( t0 : t1 )
  
    drift_weights(i) = 1 - ( i * slope_of_weights );
  
  end
\end{verbatim}
\end{Quote}

Für eine bessere Performance unter \matlab/\octave\ sollte zur Berechnung der 
Gewichte keine \cmd{for}--Schleife zum Einsatz kommen (das Beispiel oben war 
zur Verdeutlichung gedacht).

\subsubsection{$B_0$--Drift--Korrektur entlang $B_0$}

\begin{description}
  \item[Name] drift\_compensation
  \item[Funktion] Korrektur der $B_0$--Drift entlang der $B_0$--Achse
  \item[Übergabe-Parameter] 2D--Array, Anzahl der zu verwendenden 
  Dunkelsignal--Zeitkurven
  \item[Rückgabewert] 2D--Array, drift--korrigiert
\end{description}

Häufig zeigen die Spektren eine Drift über die gesamte Messung. Das ist auch 
der Grund für die symmetrische Aufnahme (s.o.). Zur Korrektur dieser Drift 
werden zwei (gleich lange) $B_0$--Intervalle am Anfang und Ende des Spektrums 
ausgewählt (\emph{per definitionem} kein Signal), gemittelt und diese 
Mittelwerte zur Drift--Korrektur verwendet.

Die beiden Korrekturwerte von Anfang und Ende des Spektrums werden gegenläufig 
gewichtet: Zu Beginn trägt nur der Mittelwert vom Anfang des Spektrums zur 
Driftkorrektur bei, sein Beitrag fällt mit fortschreitendem $B_0$ linear ab, 
gleichzeitig steigt der Beitrag des Mittelwertes vom Ende des Spektrums linear 
an, um am Ende des Spektrums sein Maximum zu erreichen.


\subsection{Fläche unter der Kurve berechnen}

\begin{description}
  \item[Name]
  \item[Funktion]
  \item[Übergabe-Parameter]
  \item[Rückgabewert]
\end{description}


\subsection{Zerfallskurve berechnen}

\begin{description}
  \item[Name]
  \item[Funktion]
  \item[Übergabe-Parameter]
  \item[Rückgabewert]
\end{description}


\section{Umsetzung der einzelnen Funktionen}

Während in den vorangegangenen Abschnitten mehr allgemeine Aspekte im 
Vordergrund standen, sollen in diesem Abschnitt konkrete umsetzungsspezifische 
Details aufgelistet werden.


\subsection{Grundlegendes}

\subsubsection{Aufteilung in einzelne Blöcke}

\matlab\ erlaubt die Deklaration von Funktionen in den sogenannten M--Files. In 
jeder solchen Datei gibt es nur \emph{eine} Haupt--Funktion und berliebig viele 
Unterfunktionen, auf die dann nur die Hauptfunktion (die an erster Stelle in 
der Datei stehen muß) zugreifen kann.


\subsubsection{Logging}

\matlab\ stellt grundsätzlich zwei Möglichkeiten bereit, alle Ausgaben in einer 
separaten Datei zu speichern: über einen Parameter beim Start von \matlab, der 
alle Ausgaben in eine dort zu definierende Datei mitschreibt, oder über die 
(deutlich komfortablere) Funktion \func{diary}, die als Parameter einen 
Dateinamen erwartet \cite[S. 124]{TheMathWorks:2005ah}:

\begin{Quote}
To create a file on your disk called \texttt{sept23.out} that contains all the 
functions you enter, as well as MATLAB output, enter
\begin{verbatim} 
diary('sept23.out')
\end{verbatim}
To stop recording the session, use
\begin{verbatim} 
diary('off')
\end{verbatim}
To view the file, run
\begin{verbatim} 
edit('sept23.out')
\end{verbatim}
\end{Quote}

Eine einfache Möglichkeit, bei jedem Funktionsaufruf die Revisionsnummer und 
das Revisionsdatum mit in der diary--Datei stehen zu haben, ist, vor dem 
jeweiligen Funktionsaufruf im Meta--Skript den Befehl ``\cmd{help 
<funktionsname>}'' aufzurufen.

Wird neben den Daten gemeinsam mit der Log--Datei das Meta--Skript abgelegt, 
sollten sich einfach und konfortabel die Verarbeitugnsschritte wiederholen und 
gegebenenfalls modifizieren lassen.


\subsubsection{Meta--Skript}

Das Meta--Skript fragt die notwendigen Parameter vom Nutzer ab und stößt die 
weitere Datenverarbeitung an. Dieses Skript kann (zumindest am Anfang) für 
jeden Datensatz angepaßt werden.

Eine eventuell zu programmierende GUI setzt ebenfalls auf dieser Ebene an. Die 
GUI ersetzt quasi das Meta--Skript. Mittlerweile sind GUI--Fähigkeiten für 
\octave\ in Entwicklung, so daß die prinzipielle Möglichkeit besteht, eine frei 
verfügbare Version der Auswertungssoftware über \octave\ zu realisieren.


\subsubsection{Versionsverwaltung}

Via CVS oder Subversion \cite{Collins-Sussman:2005qs}. Jede Funktion gibt bei 
ihrem Aufruf ihre jeweilige Revisionsnummer und nach Möglichkeit auch das Datum 
der Revision aus, um auch im Nachhinein noch nachvollziehen zu können, welche 
Version der Funktion bei der Datenverarbeitung verwendet wurde.

Wichtig ist bei der Verwendung der Versionsverwaltung die Angabe kurzer, aber 
sinnvoller und treffender Kommentare bei jedem ``commit''--Vorgang.

\matlab\ erlaubt über eingebaute Funktionen den direkten Zugriff auf 
Versionsverwaltungssysteme (und zumindest unter UNIX/Linux auch auf CVS) 
\cite[S. 431]{TheMathWorks:2005ah}. Als brauchbare Variante hat sich die 
Verwendung von \eclipse\ als Programmierumgebung herausgestellt. Es häuft unter 
allen relevanten Plattformen (UNIX/Linux, Mac, Windows) und erlaubt das 
komfortable Handling aller notwendigen CVS--Operationen.


\subsubsection{Dokumentation}

Jede \matlab\--Funktion kann durch Eingabe eines Kommentars direkt hinter der 
Funktions--Deklaration grob dokumentiert werden. Zu den unbedingt notwendigen 
Angaben gehören:

\begin{enumerate}
  \item kurze Beschreibung, was die Funktion tut
  \item Eingabe--Parameter (Anzahl, Typ)
  \item Ausgabe--Parameter (Anzahl, Typ)
\end{enumerate}

Eine ausführlichere Dokumentation sollte ebenfalls direkt im Quellcode mittels 
\robodoc\ \cite{Slothouber:2005hs} (oder einem anderen Automatisierungs--Tool 
für Dokumentationen) erfolgen.

Ein Vorteil der Kurzdokumentation in direkter Folge der Funktionsdeklaration 
ist, daß dieser Text beim Aufruf des Befehls \cmd{help routinenname} als Hilfe 
unter \matlab/\octave\ erscheint. Innerhalb dieses Blockes sollte die 
Revisionsnummer und das Revisionsdatum ausgegeben werden (für das Logging 
mittels der \cmd{diary}--Funktion).


\subsubsection{Fehlerbehandlung}

Jede Routine sollte zumindest grundlegende Fehlerbehandlungs--Fähigkeiten 
mitbringen. Dazu gehört die Überprüfung der Zahl der Eingabe-- und 
Rückgabe--Parameter ebenso wie die Überprüfung des Erfolgs beim Öffnen einer 
Datei, um anschließend in diese Datei schreiben zu können.


\subsection{Kernfunktionen}

Die Kernfunktionen sind diejenigen Funktionen, die die eigentliche 
Datenverarbeitung und --analyse bewerkstelligen. Jede dieser Funktionen wird in 
einem eigenen (gleichnamigen) M--File abgelegt und dokumentiert.

Vorrangiges Ziel ist es, die so modularisierten Funktionen zu optimieren und 
danach von der Programmentwicklung her weitestgehend einzufrieren.


\subsubsection{Einlesen der Rohdaten}

Die einfachste Variante, unter \matlab\ die von der Spektrometer--Software 
geschriebenen Daten direkt einzulesen, ist die Funktion \func{load}. Diese 
Funktion liest den Inhalt der als Parameter übergebenen Datei in eine Variable 
mit demselben Namen (ohne Dateiendung). Ein Beispiel:
%
\begin{Quote}
\begin{verbatim}
load data.dat
\end{verbatim} 
\end{Quote}
%
liest die Daten in der Datei \file{data.dat} in die Variable \var{data} ein. 
Das löst allerdings noch nicht das Problem, wie die zusätzlichen Informationen, 
die im Kommentar am Beginn der Datendatei stehen und die zur weiteren 
Verarbeitung der Daten von Interesse sind, von \matlab\ automatisiert 
eingelesen werden können.

Für das Einlesen dieser Daten stehen eine Reihe von ``low level'' 
Datei--Ein/Ausgabe--Funktionen und entsprechende 
Stringmanipulations--Funktionen zur Verfügung, vgl. Tab. \ref{tab:fileIO}.

\begin{table}
\begin{tabular*}{\textwidth}{ll}
\hline \\[-1ex] Funktion & Beschreibung\\[-1ex] \\\hline \\[-1ex] \cmd{fopen} & 
Datei öffnen\\ \cmd{fclose} & Datei schließen\\ \cmd{fgetl} & Zeile aus Datei 
einlesen, Zeilenende--Zeichen verwerfen\\[-1ex] \\\hline \\[-1ex] \cmd{findstr} 
& finde einen String in einem anderen \\[-1ex] \\\hline
\end{tabular*}
\caption{Übersicht über einige ``Low--level'' Datei--Ein/Ausgabe--Funktionen 
und String--Manipulations\-funktionen von \matlab, wie sie zum Auslesen der 
zusätzlichen Informationen aus der Datendatei verwendet werden können.}
\label{tab:fileIO}
\end{table}

Mithilfe dieser ```low level''--Funktionen wurde in der Routine
\func{read\_fsc2\_data} eine Subfunktion \cmd{read\_parameter\_from\_fsc2} 
implementiert, die als Eingabeparameter die dem Wert vorangestellte 
Beschreibung des Parameters in der \cmd{fsc2}--Datei übernimmt und den Wert des 
Parameters zurückgibt. Diese Subfunktion erlaubt auf einfache Weise, zusätzlich 
notwendig werdende Parameter aus der \cmd{fsc2}--Datei auszulesen.

Die eigentlichen Daten werden dann über den einfachsten, oben beschriebenen Weg 
(\cmd{load filename}) eingelesen.


\subsubsection{Anpassen der Reihenfolge der Daten}

Ist in der Funktion, die die Daten einliest, gleich mit implementiert worden.


\subsubsection{Akkumulation mehrerer Spektren}

Da die Spektren häufig unterschiedliche Signal--Rausch--Abstände haben, werden 
sie nicht einfach addiert, da sonst die Gefahr besteht, daß die Messungen mit 
geringem Rauschen und hohem Signal--Rausch--Abstand tendenziell schlechter 
werden.

Eine pragmatische Lösung ist, zwei Spektren gewichtet aufzuaddieren:

\begin{align*}
	\vect{C} &= \vect{A} + x \vect{B}
\end{align*}

Der Parameter $x$ wird dadurch optimiert, daß er über das Intervall $0 < x < 1$ 
variiert wird und für jeden Wert von $x$ das Verhältnis von maximaler Amplitude 
zu Rauschen bestimmt wird. $x$ gilt dann als optimiert, wenn dieses Verhältnis 
maximiert ist.

Für jede weitere aufzuaddierende Matrix wird dasselbe Verfahren angewandt.


\subsubsection{Offset--Korrektur über Pretrigger--Signal}

Die Routine mittelt für jede einzelne Zeitkurve die Daten über den Zeitbereich 
bis kurz vor dem eigentlichen Trigger--Signal (nur bis kurz vorher, um 
Artefakte durch einen extrem schnellen Signalanstieg nach dem Trigger--Impuls 
zu vermeiden) und zieht von der gesamten Zeitkurve diesen Mittelwert ab. Im 
Ergebnis erhält man ein bis auf das Rauschen auf den Nullpunkt abgebildetes 
Pretrigger--Offset.


\subsubsection{Integration eines 1D--Spektrums ($B_0$)}


%\subsubsection{Drift--Korrekur über das Dunkelsignal}
\subsubsection{$B_0$--Drift untersuchen}

Eine (über einen Parameter der Funktion übergebbare) Anzahl von $B_0$--Kurven 
am Ende der $t$--Achse des Spektrums (wo das Signal quasi zerfallen ist) wird 
gemittelt und diese so erhaltene Drift über die Funktion \func{polyfit} linear 
und quadratisch gefittet.

Die Parameter des Polynom--Fits und die zugehörigen Werte werden zusammen mit 
der Drift von der Funktion zurückgegeben und können zur Veranschaulichung und 
zur besseren Entscheidung geplottet werden.

\paragraph{TODO} Es wäre zu überlegen, inwieweit ein exponentieller Fit besser 
zur in den Daten vorliegenden Drift paßt. Dazu müßte dann eine Routine analog 
\func{poylfit} geschrieben werden, da \matlab/\octave\ standardmäßig keine 
solche Routine mitbringen.


\subsubsection{Drift--Korrekur entlang $t$}


\subsubsection{Drift--Korrekur entlang $B_0$}


\subsubsection{Berechnung der Fläche unter der Kurve}


\subsubsection{Darstellung des Zerfalls des Signals}


\bibliographystyle{abbrv}%bioscript-authors}
\bibliography{literatur}

\end{document}