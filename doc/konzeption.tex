\documentclass{article}
\usepackage{german}
\usepackage[paper=a4paper,margin=1in]{geometry}
\usepackage[utf8]{inputenc}

\usepackage{textcomp}

\title{trEPR--Messungen: Konzeption des Auswertungsprogramms}
\author{Till Biskup}
\date{\today, $ $Revision$ $}


\def\matlab{\textsf{MATLAB}$^\mbox{\tiny\textregistered}$}

\begin{document}
\maketitle\thispagestyle{empty}

\begin{abstract}
  Die folgenden Ausführungen sind eine kurze Zusammenfassung der Konzeption
  eines Auswertungsprogramms für die Daten aus transienten EPR--Messungen.
  Zunächst werden die grundlegenden Ideen für die Umsetzung kurz aufgeführt,
  anschließend die einzelnen Programmblöcke detaillierter erklärt. Ausgehend von
  dieser Dokumentation sollte es möglich sein, ein entsprechendes Programm in
  \matlab\ zu erstellen. 
\end{abstract}


\section{Grundsätzliches}

\begin{itemize}
  \item Aufteilung der einzelnen Programmblöcke in einzelne \matlab--Funktionen
  \item Dokumentation jeder einzelnen Funktion
  \item interaktive Abfrage von Parametern nicht in den Kernfunktionen, sondern
  in eigenen ``Treiber--Funktionen''
  \item Logging aller Einzelschritte und wichtigen Werte in einer Log--Datei
  \item Versionsverwaltung (CVS) der einzelnen Skripte
  \item ``Meta--Skript'', das auf die einzelnen Funktionen zurückgreift und
  einen Datensatz komplett analysiert (später durch GUI ersetzbar)
\end{itemize}


\section{Programmblöcke}

\subsection{Rohdaten einlesen}

\begin{description}
  \item[Funktion] Liest die vom Spektrometer kommenden Rohdaten (1D) ein und wandelt
  sie in ein 2D--Array um.
  \item[Übergabe-Parameter] Dateiname, $t$--Dimension, $B_0$--Dimension
  \item[Rückgabewert] 2D--Array
\end{description}


\subsection{Reihenfolge anpassen}

\begin{description}
  \item[Funktion] Ordnet die Daten (2D--Array) nach aufsteigendem $B_0$--Wert an
  \item[Übergabe-Parameter] 2D--Array
  \item[Rückgabewert] 2D--Array
\end{description}


\subsection{Akkumulation}

\begin{description}
  \item[Funktion] Aufakkumulation mehrerer Messungen
  \item[Übergabe-Parameter] mehrere 2D--Arrays mit Daten
  \item[Rückgabewert] 2D--Array mit aufakkumulierten Daten
\end{description}


\subsection{Offset--Korrektur über Pretrigger--Signal}

\begin{description}
  \item[Funktion] Offset--Korrektur des Signals über die Pretrigger--Zeit
  \item[Übergabe-Parameter] 2D--Array mit Daten, Pretrigger-Zeit
  \item[Rückgabewert] 2D--Array, offset--korrigiert
\end{description}


\subsection{Integration}

\begin{description}
  \item[Funktion]
  \item[Übergabe-Parameter] 
\end{description}


\subsection{Drift--Korrektur über Dunkelsignal}

\begin{description}
  \item[Funktion] Korrektur der Drift während der Messung über die Dunkelsignale
  \item[Übergabe-Parameter] 2D--Array, Anzahl der zu verwendenden 
  Dunkelsignal--Zeitkurven
  \item[Rückgabewert] 2D--Array, drift--korrigiert
\end{description}


\end{document}